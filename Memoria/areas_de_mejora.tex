\chapter{\'Areas de mejora}

El proyecto desarrollado es meramente un prototipo de una aplicación.
En casi todas las fases del proyecto hay aspectos que podrían
mejorarse. A continuación, exponemos algunas de ellas:
\begin{itemize} 
\item Por supuesto, implementar los recursos adecuados para que la 
LOPD no sea un obstáculo para la comercialización del proyecto.
\item Usar un acceso a Twitter que nos permita descargar de forma exhaustiva
la información en relación al perfil de referencia de la oferta de trabajo.
\item Implementar una estructura más escalable, en la nube por ejemplo,
para el manejo de un mayor volumen de datos.
\item En la fase de selección de usuarios:
\begin{itemize}
\item Detección del lenguaje: usar corpus etiquetados para entrenar un modelo,
o usar uno más adecuado para el tipo de lenguaje de los usuarios de Twitter 
(tipo {\tt equilid}).
\item Almacenamiento: usar una base de datos SQL para almacenar los resultados 
intermedios en el proceso, en la nube para mejorar la escalabilidad.
\item Tipo de usuario: explorar otras formas de detectar personas, empresas, bots, etc..
\item Naturaleza del tuit: mejorar el modelo para conseguir mayor granularidad en
la clasificación y detectar distintas especialidades dentro del perfil de referencia.
\end{itemize} 
\item En la fase de ordenación de usuarios: explotar más profundamente la información
del grafo, añadiendo características a los nodos, y usar técnicas de detección de comunidades
también para detectar especialidades.
\item En la fase de visualización, adaptar la visualización a un entorno productivo,
por ejemplo incluyendo visualización a través de una página web.
\end{itemize} 