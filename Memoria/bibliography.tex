
\begin{thebibliography}{3}
\addcontentsline{toc}{chapter}{Bibliograf\'\i a}

\bibitem{proceso_seleccion1} 
Mar\'\i a Gloria Casta\~no collado, Gerardo de la Merced L\'opez Montalvo, Jos\'e Mar\'\i a Prieto Zamora,
\textit{Gu\'\i a t\'ecnica y de buenas pr\'acticas en reclutamiento y selecci\'on de personal (R\& S)}.
Documento aprobado por la Junta de Gobierno del Colegio Oficial de Psicólogos de Madrid, Febrero de 2011.
\url{http://www.copmadrid.org/webcopm/recursos/guiatecnicabuenaspracticas.pdf}

\bibitem{proceso_seleccion2} 
\textit{Selecci\'on de personal para no especialistas}.
Andaluc\'\i a Emprende, Fundaci\'on P\'ublica Andaluza. Consejer\'\i a de Econom\'\i a y Conocimiento.
\url{https://www.andaluciaemprende.es/wp-content/uploads/2015/02/guia\-seleccion-personal.pdf}

\bibitem{lopd1} 
\textit{Ley Orgánica 15/1999, de 13 de diciembre, de Protección de Datos
de Carácter Personal}.
Jefatura del Estado BOE núm. 298, de 14 de diciembre de 1999
Referencia: BOE-A-1999-23750
\url{http://www.agpd.es/portalwebAGPD/canaldocumentacion/legislacion/estatal/common/pdfs/2014/Ley_Organica_15-1999_de_13_de_diciembre_de_Proteccion_de_Datos_Consolidado.pdf}

\bibitem{tesis_mariluz} 
María Luz Congosto Martínez,
\textit{Caracterización de usuarios y propagación de mensajes en Twitter en el entorno de temas sociales}.
Tesis doctoral.

\bibitem{twitter_wikipedia} 
``Twitter". Wikipedia. \url{https://es.wikipedia.org/wiki/Twitter}.

\bibitem{kumar_et_al}Shamanth Kumar, Fred Morstatter, Huan Liu. 
{\em  Twitter Data Analytics}. Springer (2013).

\bibitem{twitter_dev_web} Twitter Developer Documentation{\url https://dev.twitter.com/}

\bibitem{nltk_book}Steven Bird, Ewan Klein, Edward Loper. {\em Natural Language Processing with Python: 
Analyzing Text with the Natural Language Toolkit}. O'Reilly (2009). \url{http://www.nltk.org/book/}

\bibitem{zissman-berkling} Marc A. Zissman, Kay M.Berkling. Automatic language identification. {\em Speech Communication}, 
Volume 35, Issues 1–2, August 2001, Pg. 115-124.

\bibitem{almeida_estevez_piad} Y. Almeida-Cruz, S. Estévez-Velarde, A.  Piad-Morffis. Detección de Idioma en Twitter.
{\em GECONTEC: Revista Internacional de Gestión del Conocimiento y la Tecnología}, Vol.2 (3), 2014.
\url{https://www.upo.es/revistas/index.php/gecontec/article/view/1081/pdf_11}

\bibitem{langid} Marco Lui, Timothy Baldwin. {\tt langid.py}: An Off-the-shelf Language Identification Tool.
{\em Proceedings of the ACL 2012 System Demonstrations}, pg. 25--30, 2012.
\url{http://www.aclweb.org/anthology/P12-3005}

\bibitem{langid2} Marco Lui, Timothy Baldwin. Accurate Language Identification of Twitter Messages. 
{\em Proceedings of the 5th Workshop on Language Analysis for Social Media (LASM) @ EACL 2014}, pg. 17-–25,
(2014). \url{http://www.aclweb.org/anthology/W14-1303}

\bibitem{equilid} David Jurgens, Yulia Tsvetkov, Dan Jurafsky. Incorporating Dialectal Variability
for Socially Equitable Language Identification. 
{\em Proceedings of the 55th Annual Meeting of the Association for Computational Linguistics (Short Papers)}, 
pg. 51-–57, (2017). \url{https://doi.org/10.18653/v1/P17-2009}

\bibitem{libro_rrhh} Tobias M. Scholz. {\em Big Data in Organizations and the Role of
Human Resource Management}, Peter Lang Academic Research, Series: Personalmanagement und
Organisation, Vol. 5 (2017).

\bibitem{user_class1}Marco Pennacchiotti, Ana-Maria Popescu. A Machine Learning Approach to Twitter User Classification, AAAI Publications, Fifth International AAAI Conference on Weblogs and Social Media (2011).
\url{https://webpages.uncc.edu/anraja/courses/SMS/SMSBib/2886-14198-1-PB.pdf }

\bibitem{user_class2} Anjie Fang, Iadh Ounis, Philip Habel, Craig Macdonald, Nut Limsopatham.
Topic-centric Classification of Twitter User’s Political Orientation. Proceedings of the 
6th Symposium on Future Directions in Information Access (2015).
\url{http://www.dcs.gla.ac.uk/~anjie/papers/fang2015fdia.pdf }

\bibitem{user_class3} Tomoya Noro, Atsushi Mizuoka, Takehiro Tokuda. 
Towards Finding Good Twitter Users to Follow Based on User Classification.
Proceedings of The 24th International Conference on Information Modelling and Knowledge 
Bases (2014). \url{https://www.researchgate.net/publication/269994032_Towards_Finding_Good_Twitter_Users_to_Follow_Based_on_User_Classification }

\bibitem{user_class4} Zi Chu, Steven Gianvecchio, Haining Wang, Sushil Jajodia. 
Detecting Automation of Twitter Accounts: Are You a Human, Bot, or Cyborg?
IEEE Transactions on Dependable and Secure Computing, Vol. 9 (2012).
\url{http://www.cs.wm.edu/\~ hnw/paper/tdsc12b.pdf }.

\bibitem{user_class5} Yan L., Ma Q., Yoshikawa M.. Classifying Twitter Users Based on User Profile and Followers Distribution. En: Decker H., Lhotská L., Link S., Basl J., Tjoa A.M. (eds) Database and Expert Systems Applications. DEXA 2013. Lecture Notes in Computer Science, vol 8055. Springer (2013).
\url{https://link.springer.com/chapter/10.1007/978-3-642-40285-2_34 }.

\bibitem{talent_an} Jasmit Kaur, Alexis A. Fink. Trends and Practices in Talent
Analytics. SHRM-SIOP Science of HR White Paper Series (2017).
\url{http://www.siop.org/SIOP-SHRM/2017%2010_SHRM-SIOP%20Talent%20Analytics.pdf }

\bibitem{notas_fernando} Fernando  Pérez  García. Teoría de Grafos. Notas de las clases
impartidas en el Máster en Data Science, MBIT, convocatoria Marzo 2017.

\bibitem{bonacich} Phillip Bonacich, Paulette Lloyd. 
Eigenvector-like measures of centrality for asymmetric relations.
{\em Social Networks}, 23, pg. 191–201 (2001).
\url{http://citeseerx.ist.psu.edu/viewdoc/download?doi=10.1.1.226.2113&rep=rep1&type=pdf }

\bibitem{bonacich2} Phillip Bonacich. A Family of Measures. {\em American Journal 
of Sociology}, Vol. 92, No. 5, pg 1170-1182 (1987). 
\url{http://www.leonidzhukov.net/hse/2014/socialnetworks/papers/Bonacich-Centrality.pdf }.


\end{thebibliography}