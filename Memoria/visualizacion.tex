\chapter{Visualizaci\'on de los resultados}
\label{chap:visualizacion}
\section{Herramientas}
Para visualizar la tabla de los usuarios ordenados a través de su índice $h$ y de sus centralidades
hemos recurrido al paquete Shiny de R. El código con el desarrollo de esta visualización
está en la carpeta {\bf Visualizacion/Shiny} de nuestro repositorio de GitHub.

\section{Descripción de la interfaz}
Hemos considerado que los resultados mostrados al cliente serán los siguientes:
\begin{itemize}
\item Por un lado el grafo de los usuarios seleccionados, donde pinchando en cada nodo (usuario) 
muestra la información del mismo.
\item Por otro, una tabla donde cada usuario puede ser ordenado por los criterios más relevantes 
como son el índice $h$ y los diferentes tipos de centralidades. Además de ese orden sencillo 
hemos incluido una columna \lq\lq resultado\rq\rq donde ponderamos el resultado en función de los 
distintos índices que el cliente escoja como más relevantes para la selección adecuada del 
candidato. Para poder calcularlo adecuadamente hemos normalizado los datos ya que las escalas 
entre el índice $h$ y las centralidades era muy distintos. Hemos incluido también varios 
avisos para avisar al usuario en el caso de que los datos introducidos no sean correctos. 
\end{itemize}


Las definiciones de los diferentes índices se pueden obtener pasando el ratón por encima 
de los títulos de la tabla para que el cliente pueda tener a mano su significado sin necesidad 
de ir a otro sitio.


La información relativa a los usuarios siempre se da en relación a su \lq\lq user.id\rq\rq. 
En ningún caso proporcionamos nombres ni información obtenida personal directamente del análisis. 
En caso de querer contactar con ese usuario se tendría que hacer a través del propio Twitter y 
una vez obtenido el consentimiento del usuario a través de cualquier otra vía disponible 
(LinkedIn, blog, e-mail...)\footnote{Esta decisión la explicamos con detalle en la página 
\pageref{note:why_only_user_id}}. 


Del grafo cabe destacar los siguientes comentarios:
\begin{itemize}
\item Pocos nodos ($8$\%) tienen un índice $h$ igual o por encima de $1$. Teniendo en cuenta 
la definición del índice $h$ (sección \ref{subsect:indice_h}), eso significa que muchos de los 
usuarios seleccionados no tienen retuits de sus publicaciones.

\item El único usuario con un índice $h$ notable (11) tiene un in-degree de $0.039$ (no demasiado alto). 
Es decir, que este usuario no es de los que más seguidores tiene entre los usuarios destacados. Sin
embargo de sus publicaciones sobre el tema de referencia, 11 de ellas han sido retuiteadas al menos
11 veces. Además, mirando su intermediación (betweenness) tampoco es de las más altas, con lo que 
no conecta muchos de los otros usuarios entre sí. Este usuario, cuya relevancia es notable en la comunidad
general (más allá de la incluida en nuestra muestra), se captó a través de otro el cual le citaba en 
sus tuits. 


\item El usuario con mayor in-degree (mayor número de seguidores de entre los seleccionados) coincide 
con el de mayor out-degree (mayor número de amigos de entre los seleccionados). Se trata de la 
directora de una empresa del sector. 

\item El usuario con más intermediación (influencia para compartir información entre el resto de usuarios) se trata de un profesional del marketing y BI que dispone de su propio blog.


\item Varios de los usuarios con índice $h$= 1 y uno con índice $h$=2 estás \lq\lq desconectados\rq\rq de la red o casi \lq\lq desconectados\rq\rq. Esto puede ser porque el tiempo en el que duró la descarga de datos de twitter no captó al resto de sus contactos.


\item Por último mencionar que los valores tanto de \lq\lq pagerank\rq\rq como de
\lq\lq katz\_bonacich\rq\rq nunca serán igual a cero, aunque el in-degree y el out-degree del nodo 
sean cero, ya que asignan una centralidad mínima.
\end{itemize}

Para una aplicación de extensa de usuarios y relaciones, el grafo solo se mostrará para una cantidad de usuarios que aporten mayor relevancia en cualquiera de los índices.